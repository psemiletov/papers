\bibitem{afsb03}
\emph{Афанасьевский сборник. Материалы и исследования. Выпуск III, календарные обряды
и обрядовая поэзия Воронежской области}. Воронеж, 2005.

\bibitem{zakr01}
\emph{Описание Киева}. Закревский Николай. Москва, 1868.

\bibitem{zeleninrusalki}
\emph{Избранные труды. Очерки русской мифологии: Умершие неестественною смертью и русалки}. Зеленин Д. К. Индрик, Москва, 1995.

\bibitem{diakon01}
\emph{История}. Лев Диакон. Перевод М. Копыленко. Москва, 1988.

\bibitem{sinopsis}
\emph{Киевский Синопсис, или краткое собрание от различных летописей о начале славенороссийского народа}. Под редакцией Гизеля. 1836.

\bibitem{sofonovich01}
\emph{Кройника з летописцев стародавних [...]}. Софонович Ф. Киев, 1992.

\bibitem{snorry01}
\emph{Круг Земной}. Снорри Стурлусон. Москва, 1980.

\bibitem{kbagr01}
\emph{Об управлении империей.}. Константин Багрянородный. Москва, Наука, 1991, серия «Древнейшие источники по истории народов СССР».

\bibitem{zakr01}
\emph{Описание Киева}. Закревский Николай. Москва, 1868.

\bibitem{ipat}
\emph{Повесть временных лет по Ипатьевскому списку}. ПСРЛ, Том 2, Издание второе. Санкт-Петербург, 1908. 

\bibitem{sbornikmat}
\emph{Сборник материалов для исторической топографии Киева и его окрестностей, изданный Комиссией для разбора древних актов, состоящей при Киевском , Подольском и Волынском генерал-губернаторе}. Киев, 1874.

